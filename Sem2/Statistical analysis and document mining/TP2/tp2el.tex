% Options for packages loaded elsewhere
\PassOptionsToPackage{unicode}{hyperref}
\PassOptionsToPackage{hyphens}{url}
%
\documentclass[
]{article}
\usepackage{amsmath,amssymb}
\usepackage{lmodern}
\usepackage{ifxetex,ifluatex}
\ifnum 0\ifxetex 1\fi\ifluatex 1\fi=0 % if pdftex
  \usepackage[T1]{fontenc}
  \usepackage[utf8]{inputenc}
  \usepackage{textcomp} % provide euro and other symbols
\else % if luatex or xetex
  \usepackage{unicode-math}
  \defaultfontfeatures{Scale=MatchLowercase}
  \defaultfontfeatures[\rmfamily]{Ligatures=TeX,Scale=1}
\fi
% Use upquote if available, for straight quotes in verbatim environments
\IfFileExists{upquote.sty}{\usepackage{upquote}}{}
\IfFileExists{microtype.sty}{% use microtype if available
  \usepackage[]{microtype}
  \UseMicrotypeSet[protrusion]{basicmath} % disable protrusion for tt fonts
}{}
\makeatletter
\@ifundefined{KOMAClassName}{% if non-KOMA class
  \IfFileExists{parskip.sty}{%
    \usepackage{parskip}
  }{% else
    \setlength{\parindent}{0pt}
    \setlength{\parskip}{6pt plus 2pt minus 1pt}}
}{% if KOMA class
  \KOMAoptions{parskip=half}}
\makeatother
\usepackage{xcolor}
\IfFileExists{xurl.sty}{\usepackage{xurl}}{} % add URL line breaks if available
\IfFileExists{bookmark.sty}{\usepackage{bookmark}}{\usepackage{hyperref}}
\hypersetup{
  pdftitle={TP2},
  pdfauthor={Elliott Perryman, Jan Zavadil, Ignat Sabaev},
  hidelinks,
  pdfcreator={LaTeX via pandoc}}
\urlstyle{same} % disable monospaced font for URLs
\usepackage[margin=1in]{geometry}
\usepackage{color}
\usepackage{fancyvrb}
\newcommand{\VerbBar}{|}
\newcommand{\VERB}{\Verb[commandchars=\\\{\}]}
\DefineVerbatimEnvironment{Highlighting}{Verbatim}{commandchars=\\\{\}}
% Add ',fontsize=\small' for more characters per line
\usepackage{framed}
\definecolor{shadecolor}{RGB}{248,248,248}
\newenvironment{Shaded}{\begin{snugshade}}{\end{snugshade}}
\newcommand{\AlertTok}[1]{\textcolor[rgb]{0.94,0.16,0.16}{#1}}
\newcommand{\AnnotationTok}[1]{\textcolor[rgb]{0.56,0.35,0.01}{\textbf{\textit{#1}}}}
\newcommand{\AttributeTok}[1]{\textcolor[rgb]{0.77,0.63,0.00}{#1}}
\newcommand{\BaseNTok}[1]{\textcolor[rgb]{0.00,0.00,0.81}{#1}}
\newcommand{\BuiltInTok}[1]{#1}
\newcommand{\CharTok}[1]{\textcolor[rgb]{0.31,0.60,0.02}{#1}}
\newcommand{\CommentTok}[1]{\textcolor[rgb]{0.56,0.35,0.01}{\textit{#1}}}
\newcommand{\CommentVarTok}[1]{\textcolor[rgb]{0.56,0.35,0.01}{\textbf{\textit{#1}}}}
\newcommand{\ConstantTok}[1]{\textcolor[rgb]{0.00,0.00,0.00}{#1}}
\newcommand{\ControlFlowTok}[1]{\textcolor[rgb]{0.13,0.29,0.53}{\textbf{#1}}}
\newcommand{\DataTypeTok}[1]{\textcolor[rgb]{0.13,0.29,0.53}{#1}}
\newcommand{\DecValTok}[1]{\textcolor[rgb]{0.00,0.00,0.81}{#1}}
\newcommand{\DocumentationTok}[1]{\textcolor[rgb]{0.56,0.35,0.01}{\textbf{\textit{#1}}}}
\newcommand{\ErrorTok}[1]{\textcolor[rgb]{0.64,0.00,0.00}{\textbf{#1}}}
\newcommand{\ExtensionTok}[1]{#1}
\newcommand{\FloatTok}[1]{\textcolor[rgb]{0.00,0.00,0.81}{#1}}
\newcommand{\FunctionTok}[1]{\textcolor[rgb]{0.00,0.00,0.00}{#1}}
\newcommand{\ImportTok}[1]{#1}
\newcommand{\InformationTok}[1]{\textcolor[rgb]{0.56,0.35,0.01}{\textbf{\textit{#1}}}}
\newcommand{\KeywordTok}[1]{\textcolor[rgb]{0.13,0.29,0.53}{\textbf{#1}}}
\newcommand{\NormalTok}[1]{#1}
\newcommand{\OperatorTok}[1]{\textcolor[rgb]{0.81,0.36,0.00}{\textbf{#1}}}
\newcommand{\OtherTok}[1]{\textcolor[rgb]{0.56,0.35,0.01}{#1}}
\newcommand{\PreprocessorTok}[1]{\textcolor[rgb]{0.56,0.35,0.01}{\textit{#1}}}
\newcommand{\RegionMarkerTok}[1]{#1}
\newcommand{\SpecialCharTok}[1]{\textcolor[rgb]{0.00,0.00,0.00}{#1}}
\newcommand{\SpecialStringTok}[1]{\textcolor[rgb]{0.31,0.60,0.02}{#1}}
\newcommand{\StringTok}[1]{\textcolor[rgb]{0.31,0.60,0.02}{#1}}
\newcommand{\VariableTok}[1]{\textcolor[rgb]{0.00,0.00,0.00}{#1}}
\newcommand{\VerbatimStringTok}[1]{\textcolor[rgb]{0.31,0.60,0.02}{#1}}
\newcommand{\WarningTok}[1]{\textcolor[rgb]{0.56,0.35,0.01}{\textbf{\textit{#1}}}}
\usepackage{graphicx}
\makeatletter
\def\maxwidth{\ifdim\Gin@nat@width>\linewidth\linewidth\else\Gin@nat@width\fi}
\def\maxheight{\ifdim\Gin@nat@height>\textheight\textheight\else\Gin@nat@height\fi}
\makeatother
% Scale images if necessary, so that they will not overflow the page
% margins by default, and it is still possible to overwrite the defaults
% using explicit options in \includegraphics[width, height, ...]{}
\setkeys{Gin}{width=\maxwidth,height=\maxheight,keepaspectratio}
% Set default figure placement to htbp
\makeatletter
\def\fps@figure{htbp}
\makeatother
\setlength{\emergencystretch}{3em} % prevent overfull lines
\providecommand{\tightlist}{%
  \setlength{\itemsep}{0pt}\setlength{\parskip}{0pt}}
\setcounter{secnumdepth}{-\maxdimen} % remove section numbering
\ifluatex
  \usepackage{selnolig}  % disable illegal ligatures
\fi

\title{TP2}
\author{Elliott Perryman, Jan Zavadil, Ignat Sabaev}
\date{2/25/2022}

\begin{document}
\maketitle

\hypertarget{data}{%
\subsection{1 Data}\label{data}}

\begin{verbatim}
## Warning: package 'maps' was built under R version 4.1.2
\end{verbatim}

\includegraphics{tp2el_files/figure-latex/unnamed-chunk-1-1.pdf} The
code above takes the data stored in NAm2, it takes the names and
location of all the indian populations (all the individuals from a
single population have the same location in the data) and plots them
over the map of the Americas. It does so using \emph{unique} command,
which returns all the unique rows from a given vector/DataFrame.

\hypertarget{regression}{%
\subsection{2 Regression}\label{regression}}

\begin{verbatim}
## 
## Call:
## lm(formula = long ~ ., data = NAaux)
## 
## Residuals:
## ALL 494 residuals are 0: no residual degrees of freedom!
## 
## Coefficients: (5216 not defined because of singularities)
##                         Estimate Std. Error t value Pr(>|t|)
## (Intercept)            1250.3745        NaN     NaN      NaN
## L1.125                 -171.2557        NaN     NaN      NaN
## L1.130                   44.7543        NaN     NaN      NaN
## L1.135                  -54.5783        NaN     NaN      NaN
## L1.140                 -264.3330        NaN     NaN      NaN
## L1.142                 -792.4778        NaN     NaN      NaN
## L1.145                   65.6649        NaN     NaN      NaN
## L1.150                   51.0931        NaN     NaN      NaN
## L1.150.940397350993      25.7019        NaN     NaN      NaN
## L1.152                 -249.8699        NaN     NaN      NaN
## L1.155                   25.2062        NaN     NaN      NaN
## L1.160                   67.5255        NaN     NaN      NaN
##  [ reached getOption("max.print") -- omitted 5698 rows ]
## 
## Residual standard error: NaN on 0 degrees of freedom
## Multiple R-squared:      1,  Adjusted R-squared:    NaN 
## F-statistic:   NaN on 493 and 0 DF,  p-value: NA
\end{verbatim}

The linear model does not return valid results since the number of
observations is lower than the number of predictors.

\hypertarget{pca}{%
\subsection{3 PCA}\label{pca}}

\hypertarget{a}{%
\paragraph{a)}\label{a}}

Principle Component Analysis is a statistical method used to lower
dimension of observed data. It aims to represent the data in low
dimensions while keeping as much variance (and therefore information) in
the data as possible.

\hypertarget{b}{%
\paragraph{b)}\label{b}}

In this case the scaling option of \emph{prcomp} function should not be
used. All the variable are of value 0 or 1, therefore any difference in
variance between them is a valuable information which would be lost by
scaling the data.

\hypertarget{c}{%
\paragraph{c)}\label{c}}

\includegraphics{tp2el_files/figure-latex/unnamed-chunk-3-1.pdf}

\includegraphics{tp2el_files/figure-latex/unnamed-chunk-4-1.pdf}

\includegraphics{tp2el_files/figure-latex/unnamed-chunk-5-1.pdf}

\includegraphics{tp2el_files/figure-latex/unnamed-chunk-6-1.pdf}

\begin{verbatim}
## Warning: package 'fields' was built under R version 4.1.2
\end{verbatim}

\begin{verbatim}
## Loading required package: spam
\end{verbatim}

\begin{verbatim}
## Warning: package 'spam' was built under R version 4.1.2
\end{verbatim}

\begin{verbatim}
## Spam version 2.8-0 (2022-01-05) is loaded.
## Type 'help( Spam)' or 'demo( spam)' for a short introduction 
## and overview of this package.
## Help for individual functions is also obtained by adding the
## suffix '.spam' to the function name, e.g. 'help( chol.spam)'.
\end{verbatim}

\begin{verbatim}
## 
## Attaching package: 'spam'
\end{verbatim}

\begin{verbatim}
## The following objects are masked from 'package:base':
## 
##     backsolve, forwardsolve
\end{verbatim}

\begin{verbatim}
## Loading required package: viridis
\end{verbatim}

\begin{verbatim}
## Warning: package 'viridis' was built under R version 4.1.2
\end{verbatim}

\begin{verbatim}
## Loading required package: viridisLite
\end{verbatim}

\begin{verbatim}
## Warning: package 'viridisLite' was built under R version 4.1.2
\end{verbatim}

\begin{verbatim}
## 
## Attaching package: 'viridis'
\end{verbatim}

\begin{verbatim}
## The following object is masked from 'package:maps':
## 
##     unemp
\end{verbatim}

\begin{verbatim}
## 
## Try help(fields) to get started.
\end{verbatim}

\begin{Shaded}
\begin{Highlighting}[]
\FunctionTok{metric}\NormalTok{(lmlong, lmlat)}
\end{Highlighting}
\end{Shaded}

\begin{verbatim}
##          [,1]
## [1,] 1177.612
\end{verbatim}

\begin{verbatim}
## Warning in predict.lm(lmLong, X): prediction from a rank-deficient fit may be
## misleading
\end{verbatim}

\begin{verbatim}
## Warning in predict.lm(lmLat, X): prediction from a rank-deficient fit may be
## misleading
\end{verbatim}

\begin{verbatim}
## Warning in predict.lm(lmLong, X): prediction from a rank-deficient fit may be
## misleading
\end{verbatim}

\begin{verbatim}
## Warning in predict.lm(lmLat, X): prediction from a rank-deficient fit may be
## misleading
\end{verbatim}

\includegraphics{tp2el_files/figure-latex/unnamed-chunk-10-1.pdf}

\end{document}
